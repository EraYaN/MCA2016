\section{Reflection}
The design process for assignment 2 had some notable differences with the design process for assignment 1. In the core it was the same (change parameters to influence performance) but because trying out a design in assignment 2 took significantly longer than an attempt in assignment 1 simply brute forcing the best solution was a less viable option. After some effort a script was made that relatively easily could synthesize multiple designs on different cores on the same computer but it still was not fast, even on the best hardware our group had. Taking all of the above into account one of the things we learned during this assignment is that when designing systems this way you have to be cleverer than just brute force. The only reason we got any meaningful results to begin with is that before the earlier mentioned script was run a lot of possible designs were already off the table for being implausible. Then, using an iterative process, we looked at the different parameters we could change and discussed the effects of those changes so we had an idea in what direction a new round of synthesis should go.

Another thing we learned is that more is definitely not always better. Once diminishing returns kick in area and energy use go up very quickly while barely giving any performance increase in return. When designing these kind of systems you need to be able to sacrifice a couple of execution cycles if it saves a lot of area or energy. Lastly, we learned to look into more detail for performance analysis. Not just total execution cycles was evaluated. Attention was also paid to data- and instructioncache misses to determine suitable cache sizes, NOPS to determine issue width and the stop bit, and different synthesized components that determine the total area.