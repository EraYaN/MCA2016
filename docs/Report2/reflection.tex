\section{Reflection}
Assignment 2 differs significantly from assignment 1 in the sense that trying out different designs takes significantly longer. This means that in this design process we learned to be cleverer than just brute force. The only reason we got any meaningful results to begin with is that before the earlier mentioned script was run a lot of possible designs were already off the table for being implausible. Then, using an iterative process, we looked at the different parameters we could change and discussed the effects of those changes so we had an idea in what direction a new round of synthesis should go.

Another thing we learned is that more is definitely not always better. Once diminishing returns kick in area and energy use go up very quickly while barely giving any performance increase in return. When designing these kind of systems you need to be able to sacrifice a couple of execution cycles if it saves a lot of area or energy. Lastly, we learned to look into more detail for performance analysis. Not just total execution cycles was evaluated. Attention was also paid to data- and instruction cache misses to determine suitable cache sizes, NOPS to determine issue width and the stop bit, and different synthesized components that determine the total area.