\section{SoC design}
\subsection{Area estimation \& energy consumption}
For a design with optimal area and energy characteristics it's important to be able to estimate how much space elements take up on an FPGA and ultimately an ASIC.
Area wise there are some vital difference between an FPGA and an ASIC as an ASIC can directly implement functions whereas an FPGA has predefined blocks that might be utilized to implement a certain function but certainly isn't optimized.
For example: a LUT might be used to implement a NAND operation, on an ASIC this would be implemented with a NAND gate which requires far fewer transistors than a LUT.


To gain insight into area utilization we quantified the area in terms of transistors for the following four elements: registers, LUTs, RAM blocks, and DSPs/multipliers.
The first element, registers, take up very little space.
The reason being that a register can only store \num{1} bit.
Assuming a D-type flip flop is used, five transistors are required \cite{Kumar Chakravarti}.
Betz (1998) displays that a LUT consists of SRAM, multiplexers, input buffers, and complementers.
Based on these components a 4-LUTs is calculated to contain about 167 transistors.
However the FPGA used, the Xilinx XC6VLX240T, mainly utilizes 6-LUTs as is found by observing the area.txt file.
Scaling the amount of transistors to a 6-LUT amounts to about 198 transistors.
In case a logic function has to be implemented, LUTs cause quite some overhead.
Depending on the amount of inputs a logic port needs, the amount of transistors necessary to implement logic functions among the likes of NOR, NAND, and more can be range from about \numrange{2}{22} transistors (assuming a maximum of \num{10} input logic ports).
Since we think it's likely most logic ports don't go up to 10 input ports we assume an average of 4 input ports which leads to about 8 transistors on average per logic port on an ASIC.


Furthermore BRAM, which is very close to SRAM, is used on the FPGA.
A single SRAM cell requires six transistors \cite{Kumar Chakravarti}, as the FPGA has \num{18} and \SI{36}{\kibi\byte} BRAM blocks, this would mean either about \num{108000} or \num{216000} transistors are required. Lastly DSPs are a a bit harder to estimate area wise.
\cite{} however reveals that the DSP has a 25 times 18 bit two's complement multiplier.
This is of main concern for the ASIC as the DSPs are used by multipliers on the $\rho$-VEX processors.
A basic two's complement multiplier can be realized with full adders, AND gates, and NAND gates.
For a 25 times 18 bit multiplier this would require about \num{409} AND ports, 41 NAND ports, and 425 full adders.
A single full adder consists of \num{2} AND ports, \num{2} XOR ports and an OR gate.
Subsequently the amount of transistors for a multiplier can be calculated to be about \num{12818} transistors.
As a DSP contains quite some more components than just a multiplier the amount of transistors is going to be a lot higher.
Depending on the DSP this might go in to the hundreds of thousands of transistors.

The energy consumption is directly related to the amount of transistors in use.
Fortunately because the area's are estimated in the amount of transistors, it can be said that the energy consumption is directly related to the amount of transistors.
The higher the amount of transistors, the higher the energy consumption.

%Reference: http://www.ijsret.org/pdf/EATHD-15044.pdf
%Reference: https://booksonweb.files.wordpress.com/2011/11/digital-integrated-circuits-a-design-perspective-by-jan-m-rabaey.pdf

\subsection{Extra Custom Metrics}
\label{ssec:soc-design-extra-custom-metrics}
To help with selecting the best performing designs in each segment we added a couple of metrics.\\


\textbf{Energy Cycles}
This is the following equation. Thus the unit of this metric is: \si{\kilo\cycles\joule}
\begin{equation}
\frac{Cycles}{10^3} \cdot \frac{Energy}{10^3}
\end{equation}

\textbf{Area Cycles}
This is the following equation. Thus the unit of this metric is: \si{\kilo\cycles\giga\transistors}
\begin{equation}
\frac{Cycles}{10^3} \cdot \frac{Energy}{10^3} \cdot \frac{Area}{10^6}
\end{equation}

\textbf{Energy Area Cycles}
This is the following equation. Thus the unit of this metric is: \si{\kilo\cycles\joule\mega\transistors}
\begin{equation}
\frac{Cycles}{10^3} \cdot \frac{Area}{10^9}
\end{equation}

\subsection{Cache Size}
\label{ssec:soc-design-cache-size}
First we wanted to find the effect of the cache size.
In \cref{tab:cache-size-variation-results} all the results are shown.
These tests were done with a single context four lane, four multiplier configuration.
From those numbers it is clear that any data cache size over \SI{2}{\kibi\byte} is a good size.
Only \SI{8}{\kibi\byte} make a big difference, but not worth it.
The instruction cache size has a much bigger impact on the performance.
Some of our benchmarks rely heavily on the instruction cache, as shown in \cref{ssec:benchmarks-analysis}.
We see that \SI{32}{\kibi\byte} is by far the best candidate.

\begin{table}[H]
    \centering
    \caption{Cache size variation results (units same as in \cref{tab:high-performance-design-parameters})}
    \label{tab:cache-size-variation-results}
    \begin{tabular}{lSSslSSSS}
        \toprule
        \textbf{Stop Bit} & \textbf{I.Ca.}& \textbf{D.Ca.} & & \textbf{Cycles ($10^6$)} & \textbf{I. Miss (\%)} & \textbf{DR. Miss (\%)} & \textbf{DW. Miss (\%)} \\
        2 & 32 & 2 & \kibi\byte & 1.430374 &  1.14 & 1.46 & 1.53 \\
        4 & 32 & 2 & \kibi\byte & 1.463118 &  1.94 & 1.46 & 1.55 \\
        4 & 16 & 8 & \kibi\byte & 1.659030 &  4.14 & 0.24 & 0.74 \\
        4 & 16 & 2 & \kibi\byte & 1.678799 &  4.14 & 1.46 & 1.55 \\
        2 & 16 & 2 & \kibi\byte & 1.855233 &  5.33 & 1.46 & 1.53 \\
        2 &  8 & 2 & \kibi\byte & 2.014009 &  6.85 & 1.46 & 1.53 \\
        4 &  8 & 4 & \kibi\byte & 2.644193 & 14.02 & 1.37 & 1.43 \\
        4 &  8 & 2 & \kibi\byte & 2.647964 & 14.02 & 1.46 & 1.55 \\
        4 &  4 & 2 & \kibi\byte & 7.164679 & 60.88 & 1.46 & 1.56 \\
        4 &  2 & 1 & \kibi\byte & 9.885138 & 88.84 & 2.27 & 2.33 \\
        \bottomrule
    \end{tabular}
\end{table}

\subsection{High performance}
\label{ssec:soc-design-high-performance}
For our high performance design, we selected the design specified in \cref{tab:high-performance-design-parameters}. Our design is a dual core design.
\begin{table}[H]
    \caption{High performance design}
    \label{tab:high-performance-design-parameters}
    \begin{subtable}{.45\textwidth}
        \centering
        \caption{High performance design parameters}
        \begin{tabular}{lSSs}
            \toprule
            \textbf{Parameter} & \textbf{C. 0} & \textbf{C. 1} &\\
            \midrule
            Lane Configuration & {00011000} & {00011000} &\\
            Stop Bit & 2 & 2 & \\
            I. Cache Size & 16 & 32 & \kibi\byte \\
            D. Cache Size & 2 & 2 & \kibi\byte\\
            \bottomrule
        \end{tabular}
    \end{subtable}
    \quad
    \begin{subtable}{.55\textwidth}
        \centering
        \caption{High performance design performance metrics}
        \begin{tabular}{lSs}
            \toprule
            \textbf{Metric} & \textbf{Value} &\\
            \midrule
            Area & 17.071595 & \mega \\
            Cycles & 0.804621 & \mega\cycles\\
            Energy & 3.01 & \milli\joule\\
            En. Cyc. & 2.42 & \kilo\cycles\joule \\
            Area Cyc. & 13.74 & \kilo\cycles\giga\transistors \\
            En. Area Cyc. & 41.35 & \kilo\cycles\joule\mega\transistors \\
            \bottomrule
        \end{tabular}
    \end{subtable}
\end{table}

In the \cref{tab:other-high-performance-design-candidates} all candidates for the high performance design are specified, it uses the same units as \cref{tab:high-performance-design-parameters}.
All designs are ordered by their amount of execution cycles.
In the dual core design the first core runs the engine and pocsag benchmarks, the second core (with larger instruction cache) runs fir and adpcm.
This workload distribution was chosen so that each core had one large benchmark and one small benchmark to ensure that both cores finish at roughly the same time.
When a core has finished processing it is shut down using $\rho$-VEX reconfigurability in order to save power.
The second core was assigned more instruction cache memory because the fir benchmark had a high miss percentage when the instruction cache was \SI{16}{\kibi\byte} or smaller.
The high performance design in \cref{tab:high-performance-design-parameters} was chosen for it's low cycle count combined with a low energy and area use compared to the other high performance candidates.

\begin{table}[H]
    \centering
    \caption{Other high performance design candidates (units same as in \cref{tab:high-performance-design-parameters})}
    \label{tab:other-high-performance-design-candidates}
    \begin{tabular}{lllllSSSS}
        \toprule
        \# & \textbf{Lanes} & \textbf{Stop Bit} & \textbf{I. Cache}& \textbf{D. Cache} & \textbf{Area}& \textbf{Cycles} & \textbf{Energy} \\
        \midrule

        1 & 00011000 \& 00011000 &  2 \& 2 & 16 \& 32 & 2 \& 2 &  40.232925 &   0.779773  &  7.09 \\
        2 & 00011000 \& 00011000 &  2 \& 2 & 16 \& 32 & 2 \& 8 &  40.711470   &  0.778093&   8.18 \\
        3 & 00111100 \& 00111100 &  2 \& 2 & 16 \& 32 & 4 \& 4 &  42.445045   &  0.788404&   7.69 \\
        %00111100 \& 00111100 &  2 \& 2 & 32 \& 32 & 4 \& 4 &  42504185 &   7.45  &  788776
        4 & 01111110 \& 01111110 &  2 \& 2 & 16 \& 32 & 4 \& 4 &  44.585405   &  0.788884&   7.66 \\
        %01111110 \& 01111110 &  2 \& 2 & 32 \& 32 & 4 \& 4 &  44636085 &   7.2 & 788886
        5 & 0110 \& 0110  & 2 \& 2 & 16 \& 32 & 2 \& 2  & 17.071595    & 0.804621&   3.01 \\
        %01111110 \& 01111110  & 2 \& 2 &16 \& 16 &4 \& 4 &  44534185   & 8.31 &   1236457
        6 & 11111111  &  2  & 32 &2  & 23.697505  &   1.407281&  6.53 \\
        7 & 0110:0110 &  2  & 16& 2 & 23.321015  & 1.408495 &  7.90 \\
        8 & 1111 &   2  & 32 &2  & 10.467710    & 1.430374 & 2.44 \\

        \bottomrule
    \end{tabular}
\end{table}

\subsection{Balanced}
\label{ssec:soc-design-balanced}
For our balanced design, we selected the design specified in \cref{tab:balanced-design-parameters}. Our design is a dual core design.
\begin{table}[H]
    \caption{Balanced design}
    \label{tab:balanced-design-parameters}
    \begin{subtable}{.45\textwidth}
        \centering
        \caption{Balanced design parameters}
        \begin{tabular}{lSSs}
            \toprule
            \textbf{Parameter} & \textbf{C. 0} & \textbf{C. 1} &\\
            \midrule
            Lane Configuration & {0110} & {0110} &\\
            Stop Bit & 2 & 2 & \\
            I. Cache Size & 16 & 32 & \kibi\byte \\
            D. Cache Size & 2 & 2 & \kibi\byte\\
            \bottomrule
        \end{tabular}
    \end{subtable}
    \quad
    \begin{subtable}{.55\textwidth}
        \centering
        \caption{Balanced design performance metrics}
        \begin{tabular}{lSs}
            \toprule
            \textbf{Metric} & \textbf{Value} &\\
            \midrule
            Area & 17.071595 & \mega \\
            Cycles & 0.804621 & \mega\cycles\\
            Energy & 3.01 & \milli\joule\\
            En. Cyc. & 2.42 & \kilo\cycles\joule \\
            Area Cyc. & 13.74 & \kilo\cycles\giga\transistors \\
            En. Area Cyc. & 41.35 & \kilo\cycles\joule\mega\transistors \\
            \bottomrule
        \end{tabular}
    \end{subtable}
\end{table}

In the \cref{tab:other-balanced-design-candidates} all candidates for the balanced design are specified, it uses the same units as \cref{tab:balanced-design-parameters}.
We ordered all designs by their Energy Cycles (as defined in \cref{ssec:soc-design-extra-custom-metrics}).
This way we can order the designs on both Cycles and Energy at the same time.
The balanced design dual core uses the same workload distribution as discussed in \cref{ssec:soc-design-high-performance}, again with a larger instruction cache for the second core and $\rho$-VEX reconfigurability for power saving.
The balanced design was chosen for the very low energy use but still good performance in execution cycles.
The chosen design has a large area compared to the other candidates, but it was decided that this was not a problem considering a design with smaller area will be discussed in \cref{ssec:soc-design-low-power}.


\begin{table}[H]
    \centering
    \caption{Other balanced design candidates (units same as in \cref{tab:balanced-design-parameters})}
    \label{tab:other-balanced-design-candidates}
    \begin{tabular}{lllllSSSS}
        \toprule
        \# & \textbf{Lanes} & \textbf{Stop Bit} & \textbf{I. Cache}& \textbf{D. Cache} & \textbf{Area}& \textbf{Cycles} & \textbf{Energy} & \textbf{En. Cycl.}\\
        \midrule
        1 & 0110 \& 0110 & 2 \& 2 & 16 \& 32 & 2 \& 2 & 17.071595 & 0.804621 & 3.01 & 2.42 \\
        2 & 1111 & 2 & 32 & 2 & 10.467710 & 1.430374 & 2.44 & 3.49 \\
        3 & 1111 & 4 & 16 & 2 & 9.489250 & 1.678799 & 2.3 &   3.86 \\
        4 & 0110 & 4 & 16 & 2 & 8.263270 & 1.725540 & 2.3 &   3.97 \\
        5 & 1111 & 4 & 32 & 2 & 10.456770 & 1.463118 & 2.71 & 3.97 \\
        6 & 11:11 & 2 & 16 & 2 & 11.877480 & 1.505265 & 2.68 & 4.03 \\
        \bottomrule
    \end{tabular}
\end{table}

\subsection{Low power/area}
\label{ssec:soc-design-low-power}
For our low power design, we selected the design specified in \cref{tab:low-power-design-parameters}.
\begin{table}[H]
    \caption{Low power design}
    \label{tab:low-power-design-parameters}
    \begin{subtable}{.4\textwidth}
        \centering
        \caption{Low power design parameters}
        \begin{tabular}{lSs}
            \toprule
            \textbf{Parameter} & \textbf{C.}\\
            \midrule
            Lane Config & {01} &\\
            Stop Bit & 2 & \\
            Instruction Cache Size & 16 & \kibi\byte \\
            Data Cache Size & 2 & \kibi\byte \\
            \bottomrule
        \end{tabular}
    \end{subtable}
    \quad
    \begin{subtable}{.6\textwidth}
        \centering
        \caption{Low power design performance metrics}
        \begin{tabular}{lSs}
            \toprule
            \textbf{Metric} & \textbf{Value} &\\
            \midrule
            Area & 5.088640 & \mega \\
            Cycles & 1.919640 & \mega\cycles\\
            Energy & 2.71 & \milli\joule\\
            En. Cyc. & 5.20 & \kilo\cycles\joule \\
            Area Cyc. & 9.77 & \kilo\cycles\giga\transistors \\
            En. Area Cyc. & 26.47 & \kilo\cycles\joule\mega\transistors \\
            \bottomrule
        \end{tabular}
    \end{subtable}
\end{table}

In the \cref{tab:other-low-power-design-candidates} all candidates for the low power design are specified, it uses the same units as \cref{tab:low-power-design-parameters}.
We ordered all designs by their Energy Area Cycles (as defined in \cref{ssec:soc-design-extra-custom-metrics}).
This way we can order the designs on both Cycles, Area and Energy at the same time.
The chosen design is significantly slower than the other designs discussed in this section.
Unlike the balanced and high performance designs, this design is a single core so all benchmarks run sequentially.
For the low power/area design, the design shown in \cref{tab:low-power-design-parameters} still ended up being chosen when execution cycles, power and area are equally taken into account for it's very low area and power usage. 

\begin{table}[H]
    \centering
    \caption{Other low power design candidates (units same as in \cref{tab:low-power-design-parameters})}
    \label{tab:other-low-power-design-candidates}
    \begin{tabular}{lllllSSSS}
        \toprule
        \# & \textbf{Lanes} & \textbf{Stop Bit} & \textbf{I. Cache}& \textbf{D. Cache} & \textbf{Area}& \textbf{Cycles} & \textbf{Energy} & \textbf{En. Area Cycl.}\\
        \midrule
        1 & {01} & 2 & 16 & 2 & 5.088640 & 1.919640 & 2.71 & 26.47 \\
        2 & {11} & 2 & 16 & 2 & 5.699445 & 1.826294 & 3.00 & 31.23 \\
        3 & {10} & 2 & 16 & 2 & 5.195165 & 1.975335 & 3.08 & 31.61 \\
        4 & {0110} & 4 & 16 & 2 & 8.263270 & 1.725540 & 2.3 & 32.79 \\
        5 & {0110} & 2 & 16 & 2 & 8.269830 & 1.696177 & 2.48 & 34.79 \\
        \bottomrule
    \end{tabular}    
\end{table}
