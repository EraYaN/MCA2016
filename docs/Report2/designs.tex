\section{SoC design}
\subsection{Area estimation}
For a design with optimal area and energy characteristics it's important to be able to estimate how much space elements take up on an FPGA and ultimately an ASIC. Area wise there are some vital difference between an FPGA and an ASIC as an ASIC can directly implement functions whereas an FPGA has predefined blocks that might be utilized to implement a certain function but certainly isn't optimized. For example: a LUT might be used to implement a NAND operation, on an ASIC this would be implemented with a NAND gate which requires far fewer transistors than a LUT.\\\\
To gain insight into area utilization we quantified the area in terms of transistors for the following four elements: registers, LUTs, RAM blocks, and DSPs/multipliers. The first element, registers, take up very little space. The reason being that a register can only store 1 bit. Assuming a D-type flip flop is used, five transistors are required. %\cite{Kumar Chakravarti}
Betz (1998) displays that a LUT consists of SRAM, multiplexers, input buffers, and complementers. Based on these components a 4-LUTs is calculated to contain about 167 transistors. However the FPGA used, the Xilinx XC6VLX240T, mainly utilizes 6-LUTs as is found by observing the area.txt file. Scaling the amount of transistors to a 6-LUT amounts to about 198 transistors. In case a logic function has to be implemented, LUTs cause quite some overhead. Depending on the amount of inputs a logic port needs, the amount of transistors necessary to implement logic functions among the likes of NOR, NAND, and more can be range from about 2 to 22 transistors (assuming a maximum of 10 input logic ports). Since we think it's likely most logic ports don't go up to 10 input ports we assume an average of 4 input ports which leads to about 8 transistors on average per logic port on an ASIC.\\
Furthermore BRAM, which is very close to SRAM, is used on the FPGA. A single SRAM cell requires six transistors  %\cite{Kumar Chakravarti}
, as the FPGA has 18 and 36 KB BRAM blocks, this would mean either about 108.000 or 216.000 transistors are required. Lastly DSPs are a a bit harder to estimate area wise. %Reference{}
however reveals that the DSP has a 25 times 18 bit two's complement adder. This is of main concern for the ASIC as the DSPs are used by multipliers on the $\rho$-VEX processors. ...

%Reference: http://www.ijsret.org/pdf/EATHD-15044.pdf
%Reference: https://booksonweb.files.wordpress.com/2011/11/digital-integrated-circuits-a-design-perspective-by-jan-m-rabaey.pdf
\subsection{High performance}

\subsection{Balanced}
%1111 2 32k 2k
For our low power design, we selected the design specified in \cref{tab:balanced-design-parameters}.
\begin{table}[H]
    \caption{Balanced design}
    \label{tab:balanced-design-parameters}
    \begin{subtable}{.5\textwidth}
        \centering
        \caption{Balanced design parameters}
        \begin{tabular}{lSs}
            \toprule
            \textbf{Parameter} & \multicolumn{2}{c}{\textbf{Value}}\\
            \midrule
            Lane Configuration & 1111 &\\
            Stop Bit & 2 & \\
            Instruction Cache Size & 32 & \kilo\bit \\
            Data Cache Size & 2 & \kilo\bit \\
            \bottomrule
        \end{tabular}
    \end{subtable}
    \quad
    \begin{subtable}{.5\textwidth}
        \centering
        \caption{Balanced design performance metrics}
        \begin{tabular}{lSs}
            \toprule
            \textbf{Metric} & \multicolumn{2}{c}{\textbf{Value}}\\
            \midrule
            Area & 10.467710 & \mega \\
            Cycles & 1430.374 & \kilo\cycles\\
            Energy & 2.44 & \milli\joule\\
            Energy Performance & 3.49 & \kilo\cycles\joule \\
            Area Performance & 14.97 & \kilo\cycles\giga\transistors \\
            Energy Area Performance & 36.53 & \kilo\cycles\joule\mega\transistors \\
            \bottomrule
        \end{tabular}
    \end{subtable}
\end{table}

\subsection{Low power}
%01 2   16k 2k
For our low power design, we selected the design specified in \cref{tab:low-power-design-parameters}.
\begin{table}[H]
    \caption{Low power design}
    \label{tab:low-power-design-parameters}
    \begin{subtable}{.5\textwidth}
        \centering
        \caption{Low power design parameters}
        \begin{tabular}{lSs}
            \toprule
            \textbf{Parameter} & \multicolumn{2}{c}{\textbf{Value}}\\
            \midrule
            Lane Configuration & 01 &\\
            Stop Bit & 2 & \\
            Instruction Cache Size & 16 & \kilo\bit \\
            Data Cache Size & 2 & \kilo\bit \\
            \bottomrule
        \end{tabular}
    \end{subtable}
    \quad
    \begin{subtable}{.5\textwidth}
        \centering
        \caption{Low power design performance metrics}
        \begin{tabular}{lSs}
            \toprule
            \textbf{Metric} & \multicolumn{2}{c}{\textbf{Value}}\\
            \midrule
            Area & 5.088640 & \mega \\
            Cycles & 1919.640 & \kilo\cycles\\
            Energy & 2.71 & \milli\joule\\
            Energy Performance & 5.20 & \kilo\cycles\joule \\
            Area Performance & 9.77 & \kilo\cycles\giga\transistors \\
            Energy Area Performance & 26.47 & \kilo\cycles\joule\mega\transistors \\
            \bottomrule
        \end{tabular}
    \end{subtable}
\end{table}

