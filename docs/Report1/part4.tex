\section{Environment}
\label{ch:environment}
The configuration for this processor is mainly for applications where power consumption is less important that raw performance.
Think bigger embedded systems (in vehicles for example) or desktops.
Especially fir which can be used for filtering applications.
When you need said function in a low power environment you will use a special function unit on the target device (DSP).
The functionality of engine would be use close to an actual engine, thus power is not a problem.
Most modern alternators deliver plenty of power to run even the biggest desktop size chips.
The trend in cars is towards more powerful hardware to enable all kinds of smart features.

The best performance was achieved using a single relatively big single cluster that is not terribly power efficient, but achieves high performance.
That especially in for example automotive can be crucial to bring latencies down and safety up.

\section{Conclusion}
A configuration for a processor was developed that not only significantly improves the performance of both benchmarks but also manages to stay relatively small. For engine the amount of execution cycles was brought down from 724703 to 521269 and fir from 509595 down to 472448. Improvements of respectively 29.7\% and 9.4\%. The improvement is significantly better for engine. Attempts at finding a configuration that had as shocking an improvement for fir as for engine were unsuccessful, though not for lack of trying. Of course, not every program leaves a lot of room for improvement. Some processes are inherently more sequential and therefore hard to improve by doing more work at the same time.

It should be noted that while developing this configuration a lot of priority was given to keeping the processor somewhat small and efficient. Possibly more so than was necessary for the environment laid out in \cref{ch:environment}. A slightly bigger performance improvement was still a possibility. Using an issue width of 8 with 6 ALU's and 4 multiply blocks yielded execution cycles of 517629 and 471082 for respectively engine and fir for example. Despite that, it was decided that the improvements weren't big enough to warrant an upgrade in size, cost and power consumption.

\section{Reflection}
This assignment has given us a start point for familiarizing ourselves with VLIW processors. We've been taught to first dissect a program to identify the most common operations so we can configure our processor to suit the program. We've learned how to use and interpret some useful profiling tools. Subsequently we had to start thinking about relevant properties like the amount of ALU syllables processed per cycle per cluster. Or about the amount of clusters we would actually like to use. Furthermore we've now become more familiar with the Linux distro, compiling, and editing the configuration file. This should have prepared us well for the final lab assignment.