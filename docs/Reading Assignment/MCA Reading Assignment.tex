
%% bare_conf.tex
%% V1.4
%% 2012/12/27
%% by Michael Shell
%% See:
%% http://www.michaelshell.org/
%% for current contact information.
%%
%% This is a skeleton file demonstrating the use of IEEEtran.cls
%% (requires IEEEtran.cls version 1.8 or later) with an IEEE conference paper.
%%
%% Support sites:
%% http://www.michaelshell.org/tex/ieeetran/
%% http://www.ctan.org/tex-archive/macros/latex/contrib/IEEEtran/
%% and
%% http://www.ieee.org/

%%*************************************************************************
%% Legal Notice:
%% This code is offered as-is without any warranty either expressed or
%% implied; without even the implied warranty of MERCHANTABILITY or
%% FITNESS FOR A PARTICULAR PURPOSE! 
%% User assumes all risk.
%% In no event shall IEEE or any contributor to this code be liable for
%% any damages or losses, including, but not limited to, incidental,
%% consequential, or any other damages, resulting from the use or misuse
%% of any information contained here.
%%
%% All comments are the opinions of their respective authors and are not
%% necessarily endorsed by the IEEE.
%%
%% This work is distributed under the LaTeX Project Public License (LPPL)
%% ( http://www.latex-project.org/ ) version 1.3, and may be freely used,
%% distributed and modified. A copy of the LPPL, version 1.3, is included
%% in the base LaTeX documentation of all distributions of LaTeX released
%% 2003/12/01 or later.
%% Retain all contribution notices and credits.
%% ** Modified files should be clearly indicated as such, including  **
%% ** renaming them and changing author support contact information. **
%%
%% File list of work: IEEEtran.cls, IEEEtran_HOWTO.pdf, bare_adv.tex,
%%                    bare_conf.tex, bare_jrnl.tex, bare_jrnl_compsoc.tex,
%%                    bare_jrnl_transmag.tex
%%*************************************************************************

% *** Authors should verify (and, if needed, correct) their LaTeX system  ***
% *** with the testflow diagnostic prior to trusting their LaTeX platform ***
% *** with production work. IEEE's font choices can trigger bugs that do  ***
% *** not appear when using other class files.                            ***
% The testflow support page is at:
% http://www.michaelshell.org/tex/testflow/



% Note that the a4paper option is mainly intended so that authors in
% countries using A4 can easily print to A4 and see how their papers will
% look in print - the typesetting of the document will not typically be
% affected with changes in paper size (but the bottom and side margins will).
% Use the testflow package mentioned above to verify correct handling of
% both paper sizes by the user's LaTeX system.
%
% Also note that the "draftcls" or "draftclsnofoot", not "draft", option
% should be used if it is desired that the figures are to be displayed in
% draft mode.
%
\documentclass[conference]{IEEEtran}
% Add the compsoc option for Computer Society conferences.
%
% If IEEEtran.cls has not been installed into the LaTeX system files,
% manually specify the path to it like:
% \documentclass[conference]{../sty/IEEEtran}





% Some very useful LaTeX packages include:
% (uncomment the ones you want to load)


% *** MISC UTILITY PACKAGES ***
%
%\usepackage{ifpdf}
% Heiko Oberdiek's ifpdf.sty is very useful if you need conditional
% compilation based on whether the output is pdf or dvi.
% usage:
% \ifpdf
%   % pdf code
% \else
%   % dvi code
% \fi
% The latest version of ifpdf.sty can be obtained from:
% http://www.ctan.org/tex-archive/macros/latex/contrib/oberdiek/
% Also, note that IEEEtran.cls V1.7 and later provides a builtin
% \ifCLASSINFOpdf conditional that works the same way.
% When switching from latex to pdflatex and vice-versa, the compiler may
% have to be run twice to clear warning/error messages.






% *** CITATION PACKAGES ***
%
%\usepackage{cite}
% cite.sty was written by Donald Arseneau
% V1.6 and later of IEEEtran pre-defines the format of the cite.sty package
% \cite{} output to follow that of IEEE. Loading the cite package will
% result in citation numbers being automatically sorted and properly
% "compressed/ranged". e.g., [1], [9], [2], [7], [5], [6] without using
% cite.sty will become [1], [2], [5]--[7], [9] using cite.sty. cite.sty's
% \cite will automatically add leading space, if needed. Use cite.sty's
% noadjust option (cite.sty V3.8 and later) if you want to turn this off
% such as if a citation ever needs to be enclosed in parenthesis.
% cite.sty is already installed on most LaTeX systems. Be sure and use
% version 4.0 (2003-05-27) and later if using hyperref.sty. cite.sty does
% not currently provide for hyperlinked citations.
% The latest version can be obtained at:
% http://www.ctan.org/tex-archive/macros/latex/contrib/cite/
% The documentation is contained in the cite.sty file itself.






% *** GRAPHICS RELATED PACKAGES ***
%
\ifCLASSINFOpdf
  % \usepackage[pdftex]{graphicx}
  % declare the path(s) where your graphic files are
  % \graphicspath{{../pdf/}{../jpeg/}}
  % and their extensions so you won't have to specify these with
  % every instance of \includegraphics
  % \DeclareGraphicsExtensions{.pdf,.jpeg,.png}
\else
  % or other class option (dvipsone, dvipdf, if not using dvips). graphicx
  % will default to the driver specified in the system graphics.cfg if no
  % driver is specified.
  % \usepackage[dvips]{graphicx}
  % declare the path(s) where your graphic files are
  % \graphicspath{{../eps/}}
  % and their extensions so you won't have to specify these with
  % every instance of \includegraphics
  % \DeclareGraphicsExtensions{.eps}
\fi
% graphicx was written by David Carlisle and Sebastian Rahtz. It is
% required if you want graphics, photos, etc. graphicx.sty is already
% installed on most LaTeX systems. The latest version and documentation
% can be obtained at: 
% http://www.ctan.org/tex-archive/macros/latex/required/graphics/
% Another good source of documentation is "Using Imported Graphics in
% LaTeX2e" by Keith Reckdahl which can be found at:
% http://www.ctan.org/tex-archive/info/epslatex/
%
% latex, and pdflatex in dvi mode, support graphics in encapsulated
% postscript (.eps) format. pdflatex in pdf mode supports graphics
% in .pdf, .jpeg, .png and .mps (metapost) formats. Users should ensure
% that all non-photo figures use a vector format (.eps, .pdf, .mps) and
% not a bitmapped formats (.jpeg, .png). IEEE frowns on bitmapped formats
% which can result in "jaggedy"/blurry rendering of lines and letters as
% well as large increases in file sizes.
%
% You can find documentation about the pdfTeX application at:
% http://www.tug.org/applications/pdftex





% *** MATH PACKAGES ***
%
%\usepackage[cmex10]{amsmath}
% A popular package from the American Mathematical Society that provides
% many useful and powerful commands for dealing with mathematics. If using
% it, be sure to load this package with the cmex10 option to ensure that
% only type 1 fonts will utilized at all point sizes. Without this option,
% it is possible that some math symbols, particularly those within
% footnotes, will be rendered in bitmap form which will result in a
% document that can not be IEEE Xplore compliant!
%
% Also, note that the amsmath package sets \interdisplaylinepenalty to 10000
% thus preventing page breaks from occurring within multiline equations. Use:
%\interdisplaylinepenalty=2500
% after loading amsmath to restore such page breaks as IEEEtran.cls normally
% does. amsmath.sty is already installed on most LaTeX systems. The latest
% version and documentation can be obtained at:
% http://www.ctan.org/tex-archive/macros/latex/required/amslatex/math/





% *** SPECIALIZED LIST PACKAGES ***
%
%\usepackage{algorithmic}
% algorithmic.sty was written by Peter Williams and Rogerio Brito.
% This package provides an algorithmic environment fo describing algorithms.
% You can use the algorithmic environment in-text or within a figure
% environment to provide for a floating algorithm. Do NOT use the algorithm
% floating environment provided by algorithm.sty (by the same authors) or
% algorithm2e.sty (by Christophe Fiorio) as IEEE does not use dedicated
% algorithm float types and packages that provide these will not provide
% correct IEEE style captions. The latest version and documentation of
% algorithmic.sty can be obtained at:
% http://www.ctan.org/tex-archive/macros/latex/contrib/algorithms/
% There is also a support site at:
% http://algorithms.berlios.de/index.html
% Also of interest may be the (relatively newer and more customizable)
% algorithmicx.sty package by Szasz Janos:
% http://www.ctan.org/tex-archive/macros/latex/contrib/algorithmicx/




% *** ALIGNMENT PACKAGES ***
%
%\usepackage{array}
% Frank Mittelbach's and David Carlisle's array.sty patches and improves
% the standard LaTeX2e array and tabular environments to provide better
% appearance and additional user controls. As the default LaTeX2e table
% generation code is lacking to the point of almost being broken with
% respect to the quality of the end results, all users are strongly
% advised to use an enhanced (at the very least that provided by array.sty)
% set of table tools. array.sty is already installed on most systems. The
% latest version and documentation can be obtained at:
% http://www.ctan.org/tex-archive/macros/latex/required/tools/


% IEEEtran contains the IEEEeqnarray family of commands that can be used to
% generate multiline equations as well as matrices, tables, etc., of high
% quality.




% *** SUBFIGURE PACKAGES ***
%\ifCLASSOPTIONcompsoc
%  \usepackage[caption=false,font=normalsize,labelfont=sf,textfont=sf]{subfig}
%\else
%  \usepackage[caption=false,font=footnotesize]{subfig}
%\fi
% subfig.sty, written by Steven Douglas Cochran, is the modern replacement
% for subfigure.sty, the latter of which is no longer maintained and is
% incompatible with some LaTeX packages including fixltx2e. However,
% subfig.sty requires and automatically loads Axel Sommerfeldt's caption.sty
% which will override IEEEtran.cls' handling of captions and this will result
% in non-IEEE style figure/table captions. To prevent this problem, be sure
% and invoke subfig.sty's "caption=false" package option (available since
% subfig.sty version 1.3, 2005/06/28) as this is will preserve IEEEtran.cls
% handling of captions.
% Note that the Computer Society format requires a larger sans serif font
% than the serif footnote size font used in traditional IEEE formatting
% and thus the need to invoke different subfig.sty package options depending
% on whether compsoc mode has been enabled.
%
% The latest version and documentation of subfig.sty can be obtained at:
% http://www.ctan.org/tex-archive/macros/latex/contrib/subfig/




% *** FLOAT PACKAGES ***
%
%\usepackage{fixltx2e}
% fixltx2e, the successor to the earlier fix2col.sty, was written by
% Frank Mittelbach and David Carlisle. This package corrects a few problems
% in the LaTeX2e kernel, the most notable of which is that in current
% LaTeX2e releases, the ordering of single and double column floats is not
% guaranteed to be preserved. Thus, an unpatched LaTeX2e can allow a
% single column figure to be placed prior to an earlier double column
% figure. The latest version and documentation can be found at:
% http://www.ctan.org/tex-archive/macros/latex/base/


%\usepackage{stfloats}
% stfloats.sty was written by Sigitas Tolusis. This package gives LaTeX2e
% the ability to do double column floats at the bottom of the page as well
% as the top. (e.g., "\begin{figure*}[!b]" is not normally possible in
% LaTeX2e). It also provides a command:
%\fnbelowfloat
% to enable the placement of footnotes below bottom floats (the standard
% LaTeX2e kernel puts them above bottom floats). This is an invasive package
% which rewrites many portions of the LaTeX2e float routines. It may not work
% with other packages that modify the LaTeX2e float routines. The latest
% version and documentation can be obtained at:
% http://www.ctan.org/tex-archive/macros/latex/contrib/sttools/
% Do not use the stfloats baselinefloat ability as IEEE does not allow
% \baselineskip to stretch. Authors submitting work to the IEEE should note
% that IEEE rarely uses double column equations and that authors should try
% to avoid such use. Do not be tempted to use the cuted.sty or midfloat.sty
% packages (also by Sigitas Tolusis) as IEEE does not format its papers in
% such ways.
% Do not attempt to use stfloats with fixltx2e as they are incompatible.
% Instead, use Morten Hogholm'a dblfloatfix which combines the features
% of both fixltx2e and stfloats:
%
% \usepackage{dblfloatfix}
% The latest version can be found at:
% http://www.ctan.org/tex-archive/macros/latex/contrib/dblfloatfix/
\usepackage{dirtytalk}
\usepackage{todonotes}




% *** PDF, URL AND HYPERLINK PACKAGES ***
%
%\usepackage{url}
% url.sty was written by Donald Arseneau. It provides better support for
% handling and breaking URLs. url.sty is already installed on most LaTeX
% systems. The latest version and documentation can be obtained at:
% http://www.ctan.org/tex-archive/macros/latex/contrib/url/
% Basically, \url{my_url_here}.




% *** Do not adjust lengths that control margins, column widths, etc. ***
% *** Do not use packages that alter fonts (such as pslatex).         ***
% There should be no need to do such things with IEEEtran.cls V1.6 and later.
% (Unless specifically asked to do so by the journal or conference you plan
% to submit to, of course. )


% correct bad hyphenation here
\hyphenation{op-tical net-works semi-conduc-tor}


\begin{document}
%
% paper title
% can use linebreaks \\ within to get better formatting as desired
% Do not put math or special symbols in the title.
\title{A Comparison of Branch Prediction Strategies}


% author names and affiliations
% use a multiple column layout for up to three different
% affiliations
\author{\IEEEauthorblockN{E.R. de Haan, A.S.J. Misdorp, L.T.J. van Leeuwen}
\IEEEauthorblockA{Computer Engineering Laboratory\\
Delft University of Technology\\
Delft, The Netherlands\\
Email: e.r.dehaan@student.tudelft.nl, a.s.j.misdorp@student.tudelft.nl, l.t.j.vanleeuwen@student.tudelft.nl}
%\and
%\IEEEauthorblockN{Homer Simpson}
%\IEEEauthorblockA{Twentieth Century Fox\\
%Springfield, USA\\
%Email: homer@thesimpsons.com}
%\and
%\IEEEauthorblockN{James Kirk\\ and Montgomery Scott}
%\IEEEauthorblockA{Starfleet Academy\\
%San Francisco, California 96678-2391\\
%Telephone: (800) 555--1212\\
%Fax: (888) 555--1212}}
% conference papers do not typically use \thanks and this command
% is locked out in conference mode. If really needed, such as for
% the acknowledgment of grants, issue a \IEEEoverridecommandlockouts
% after \documentclass

% for over three affiliations, or if they all won't fit within the width
% of the page, use this alternative format:
% 
%\author{\IEEEauthorblockN{Michael Shell\IEEEauthorrefmark{1},
%Homer Simpson\IEEEauthorrefmark{2},
%James Kirk\IEEEauthorrefmark{3}, 
%Montgomery Scott\IEEEauthorrefmark{3} and
%Eldon Tyrell\IEEEauthorrefmark{4}}
%\IEEEauthorblockA{\IEEEauthorrefmark{1}School of Electrical and Computer Engineering\\
%Georgia Institute of Technology,
%Atlanta, Georgia 30332--0250\\ Email: see http://www.michaelshell.org/contact.html}
%\IEEEauthorblockA{\IEEEauthorrefmark{2}Twentieth Century Fox, Springfield, USA\\
%Email: homer@thesimpsons.com}
%\IEEEauthorblockA{\IEEEauthorrefmark{3}Starfleet Academy, San Francisco, California 96678-2391\\
%Telephone: (800) 555--1212, Fax: (888) 555--1212}
%\IEEEauthorblockA{\IEEEauthorrefmark{4}Tyrell Inc., 123 Replicant Street, Los Angeles, California 90210--4321}
}




% use for special paper notices
%\IEEEspecialpapernotice{(Invited Paper)}




% make the title area
\maketitle

% As a general rule, do not put math, special symbols or citations
% in the abstract
\textbf{\textit{Abstract}-- This paper discusses the workings of four different branch prediction techniques in detail: two-level adaptive training, hybrid prediction, YAGS and neural branch prediction. Their accuracies are compared for different predictor sizes using data from the papers that originally suggested the prediction schemes. A direct comparison is made difficult by the fact that the different papers use different benchmarks and data collection methods. If that problem is ignored hybrid prediction would perform the best for large predictor sizes with 98.1\% accuracy.}


\IEEEpeerreviewmaketitle



\section{Introduction}

Any arbitrary computer program utilizes several conditional statements like the \textit{if-else} construction and \textit{for} or \textit{while} loops.
These conditional statements can prove to be disruptive to the program flow as an instruction is usually fetched from the program counter, after which the next instruction is usually assumed to be at the next address.
As current computer architectures employ pipelines, it is preferable to fetch the next instruction while the previous one is being executed.
However with conditional statements it's uncertain what the next instruction will be.
In the case of an \textit{if-else} statement there are two possibilities, both which can only be evaluated once this statement is executed.
Normally this would mean that the pipeline would have to stall and wait for the statement to be evaluated.
Branch predictors try to minimize the delay caused by potential stalling due to conditional statements.
There are various ways to accomplish this, four of those known methods are: the hybrid predictor, neural branch prediction, two-level adaptive training, and the YAGS branch prediction.
These four methods will be described and analyzed after which a comparison will be drawn.
With this comparison we would like to answer the following research question:
\enquote{Which of the four proposed branch predictors provide the highest accuracy, while requiring the least amount of resources?}

In order to provide a good answer to this question the workings of each of the analysed branch predictors will first be explained. 
After that the performance of each of the methods will be evaluated and compared with each other. 
In the end it will be found that the hybrid predictor has the highest accuracy with it's paper claiming that 98.1\% of it's predictions are correct. 
It should be noted though that the comparison has been made very difficult because of varying benchmarks and data collection methods between the different papers.
%\begin{itemize}
%\item Introduce the topic and its relevance to computer architecture.
%\item State the Research Question (RQ).
%\item Briefly describe the RQ solutions to be detailed in the report.
%\item State the main conclusions of your investigation.
%\item Report organisation (one paragraph).
%\end{itemize} 


\section{Description of the Evaluated Solutions}
In order of the year the respective papers were published they will be discussed. The proposed branch prediction schemes of each paper will be explained to serve as context for the quantitative and qualitative comparison that follows this section. Attention will also be paid to general branch prediction concepts, especially in the older papers.

\subsection{Two-level adaptive training}
\label{two-level}
The first paper that will be discussed is Two-Level Adaptive Training Branch Prediction by Tse-Yu Yeh and Yale N.
Patt \cite{twolevel}.
This paper is the oldest of the four featured in this comparison paper, dating from 1991.
The ideas put forth in this paper are however influential enough to be referred to by two of the other four papers and it is therefore definitely worth discussing.
Two-level adaptive training is a local branch prediction scheme.
The great benefit of local branch prediction is that it is able to remember the histories of multiple conditional branches separately.
This means that the taken or not taken choices of one branch do not necessarily influence the predictions made for other branches in the code.
The two-level adaptive training scheme accomplishes this by using two separate registers: the branch history register table (HRT) and the branch history pattern table (PT).
\subsubsection{Mechanism}
Two-level adaptive training uses pattern repetition to predict the outcome of branches.
The job of the HRT is to save the recent history of choices made in branches.
Ideally every branch has its own register in the table but considering the amount of conditional branches that can be present in a program this is obviously not the case (which will be discussed later).
When a branch is taken a '1' is shifted into the register and if a branch is not taken a '0' will be shifted in.
The contents of the registers in the HRT are used to address the PT.
In that table the actual prediction bits reside.
In order to have separate prediction bits for every possible pattern, the PT will have to have $2^k$ entries where k is the amount of bits in the history registers in the HRT.
The PT has multiple possible schemes it can use to produce predictions for the occurring patterns.
A simple example would be a "last time" scheme where the pattern table simply saves the outcome of the most recent time this pattern occurred and predicts that the outcome will probably be the same this time.
Yeh and Patt favor a saturating counter in their paper as it seemed to perform the best.
This saturating counter is a simple two bit counter that increments whenever a branch is taken and decrements when a branch is not taken.
Saturating means that it won't overflow or underflow past 3 or 0.
The prediction of this saturating counter is then based on it's most significant bit, predicting taken if it is '1' and not taken if it is '0'.
Using such a counter introduces a form inertia in the predictions, meaning that one wrong prediction won't immediately change the next prediction.
Tables of these saturating counters are often called bimodal branch predictors.
\subsubsection{Possible problems}
With a finite amount of resources two-level adaptive training inevitably runs into some limits.
As mentioned before, having an HRT large enough to store a separate history for every single conditional branch in the program is infeasible.
Yeh and Patt suggest two ways of making the two-level adaptive training work with smaller HRT's.
The Associative History Register Table (AHRT) and the Hash History Register Table (HHRT).
The AHRT is a simple set-associative cache where the lower part of a branch address is used to index the table and the higher part is used as a tag.
A least recently used algorithm is used to determine what branch histories are saved.
The AHRT comes with the extra cost of having to store the tag.
The HHRT is a simple hash table so it has the benefit of not having to store the tag, but it has a slightly worse performance because collisions can occur in the table.
This means that the histories of two different branches can interfere with each other.

Another problem is that the two-level adaptive training scheme utilizes only one PT.
This means that if a certain pattern occurs in two different branches with a different outcome predictions will be flawed because the two patterns will address the same prediction bits in the PT.
This interference effect can be reduced by saving a longer history of every branch.

Lastly, there are some possible problems with latency.
Because making a prediction requires two sequential lookups in two different tables, the scheme is somewhat slow.
In order to not have to wait for a prediction Yeh and Patt suggest already determining the next prediction whenever the history register is updated and storing it in the history register for that branch.
This makes the prediction immediately available whenever the HRT is accessed.
Issues can also occur when a new prediction in a branch is required before the last one has been confirmed to be correct or incorrect.
Yeh and Patt suggest to just predict that the branch is taken in such cases as this mostly occurs in very tight loops with a high tendency to be taken.

\subsection{Hybrid predictor}
The second paper that will be discussed is "Combining Branch Predictors" by Scott McFarling \cite{hybrid}.
With his idea for a combined predictor McFarling combines older branch prediction concepts into a new form with increased accuracy.
Before McFarling goes into detail for his hybrid predictor he first discusses some basic branch prediction concepts of which some are already discussed in the last section.
Recall, bimodal prediction is the use of saturating counters to make predictions that don't immediately change upon an incorrect choice and local prediction is the idea that one saves separate histories for different conditional branches for more accurate case specific predictions.
The local prediction scheme McFarling mentions is in fact the very same two-level adaptive training scheme discussed in the last section.
In order to gain a better understanding of McFarling's predictor, global branch prediction will briefly have to be explained.
\subsubsection{Global branch prediction}
This concept does away with the history for different branches that local branch prediction utilizes and instead looks at the recent history of the program itself.
It saves the choice made whenever a branch is encountered without care for what branch it is and then uses the resulting pattern to address a register of counters.
This of course becomes increasingly accurate with a longer saved history for choices.
The paper shows that global branch prediction is worse than local for all register sizes, but better than bimodal for larger register sizes.
McFarling also discusses in his paper an expansion on this concept that introduces some locality in the prediction to increase the performance.
By concatenating parts of (called gselect in the paper) or XORing (called gshare) the branch address and global history and using that to address a register of counters significantly improved performance is achieved for all register sizes.
Doing this essentially reintroduces a certain degree of locality to the prediction scheme as the branch address start influencing the prediction again.
Global branch prediction can be especially effective when the outcome of a branch depends on the outcome of other recent branches.
\subsubsection{Hybrid branch prediction}
The most important conclusion of McFarling's paper and the reason it's included in this comparison paper is his suggestion for a hybrid predictor.
The hybrid predictor functions as a sort of meta-predictor.
It combines the concepts mentioned above as two predictors in one and then predicts which of the two is most likely to give a correct prediction.
In order to accomplish this it needs an extra register of counters addressed by the program counter.
Much in the same way the bimodal branch predictor uses its counter register it increments or decrements the 2 bit counters in the table depending on which of the two predictors was right and uses the most significant bit in the counter to determine which predictor to trust.
Of course, the hybrid predictor allows for multiple combinations of different predictors.
In the paper however a focus is put on combining a bimodal or local predictor with a gshare predictor.
Using either of these combinations results in higher prediction accuracies than achieved by any of the earlier discussed methods.
Using the local/gshare combination seems to obtain the highest achievable accuracy with arrays larger than 2KB approaching 98.1\% accuracy.
%Describe the main idea of the solution in [1] and the way it works. Use your own words and in case really needed you can include figures from the original paper while giving proper reference. The text has to be self contained as the reader should be able to understand the essence of the proposed approach without reading the original paper.

\subsection{YAGS predictor}
\label{ssec:yags}
The third paper that will be discussed is "The YAGS Branch Prediction Scheme" by A. N. Eden and T. Mudge \cite{yags}.
The YAGS predictor is a bimodal predictor at it's core.
It should combat aliasing between biased branches and their instances where they do not agree with their bias.
And lastly it would be beneficial to reduce the amount of unnecessary information in the PHT.

So if the bimodal predictor is used to store the branch bias, one only needs to store the instances of the branches when they do not agree with their bias in the direction PHTs.
This reduces the size of the direction PHTs greatly.
To keep track of those instances a 6 to 8 bit tag is added to the direction PHTs, these tags are based on the least significant bits of the branch address, and so almost all aliasing between two consecutive branches is eliminated.

So when a branch occurs, and the chosen PHT indicates a taken bias, the not taken cache is checked to see if there is a cache hit.
If this is the case the cache provides the prediction, otherwise the chosen PHT.
The direction caches are updated if one of their predictions was used, and the chosen PHT is updated like in a bimodal predictor.
The last way the one of the direction caches is updated is if the chosen PHT's supplied prediction turned out to be wrong, the cache of the correct result is updated, because the branch did not agree with it's bias there.

A last improvement is to make the direction caches set-associative, to reduce the aliasing even further.
A LRU (least recently used) replacement policy is used, with the exception that an entry in the taken cache indicating not taken will be replaced first to avoid unnecessary information, this information is already in the choice PHT and therefore redundant.

\subsection{Neural Branch prediction}

In 2003 Jiminez \cite{neural} proposed a modification of the standard neural branch predictors called \enqoute{Fast Path-Based Neural Branch Prediction}. The main reason for this modification is that a regular neural branch predictor has high latency. Jiminez managed to decrease this latency by spreading the work required for neural branch prediction over time. First a brief introduction to neural branch prediction is given after which the Fast Path-based version is described.\\

Neural branch prediction makes use of perceptrons. A perceptron is based on a neuron which has dendrites to receive information, a cell body to process this information, and axon terminals to relay the output. A perceptron is highly similar in the sense that it has inputs, a processing stage, and an output. The inputs of a perceptron take in previous branch result that represent either taken or not taken. These branch results are stored in a history register. Every input of a perceptron is weighted, this way a potential correlation between inputs can be indicated.  When a branch prediction has to be made, the branch address is hashed to select a perceptron from a table. Subsequently the dot product of the input, the history register, and the corresponding integer weights of the perceptron is taken. This outputs a single integer that represents the output of the perceptron. A positive output indicates that the branch should be taken, while a negative output indicates the branch shouldn't be taken. The key to successful prediction is that the perceptrons are trained. Once the branch result is know the perceptron is confronted with its prediction from which it can learn. Subsequently the perceptron's weights are updated to improve the prediction.

\subsubsection{Fast Path-Based Neural Branch Prediction}
The main problem with regular neural branch prediction is latency. Normally a perceptron is selected based on the hash of the branch address after which the output is calculated as the dot product of the weights and the history vector. A path-based algorithm works differently in that the weights are chosen along the path to the branch. Take for example a regular perceptron predictor with six weights that are used to predict branch $B_t$. Instead of taking the dot product of all the weights with the history register at once, five of the six weights are added to its corresponding value in the history register during the prediction of the previous branches $B_6$ until $B_1$. These additions are summed and during the prediction of branch $B_t$, the last weight and history register addition is performed and added to the sum. This method where the computation is spread out over time has lead to a 16\% higher IPC on a predictor with 64 KB hardware. Another positive effect is that this method increases accuracy as the path history is taken into account. Lastly the weights can be updated again according to the actual branch result.
 

% metric is SPEC95 branch prediction correctness (between 0-1) Quantitative
% metric is size on-chip Qualitative

\section{Comparison}
\subsection{Metrics}
As a comparison metric the branch prediction accuracy is the most obvious.
For all of the talked about solutions there are figures for SPEC benchmarks.

A second good metric is the size of each of the solutions and their relative performance.
\subsection{Results}

%TODO table to SPEC (Quantative)

%TODO piece of text for size.


% Erwin info:
Combined:
The branch predictor was benchmarked using SPEC 89, running the programs for 10 million instructions. For large predictor sizes (around 64 KB) the accuracy of the best performing predictor combination (local/gshare) approaches 98.1\%. This is better than either of the two alone. Of course, the predictor becomes less accurate once the array sizes are shrunk. A great advantage of the combined predictor is that it is very adaptable. There are a lot of combinations of predictors that could possibly be implemented and if resources are at a premium the predictor can easily be shrunk at the cost of performance.

Two-level:
The oldest branch predictor of this paper unsurprisingly does not have the best performance. On nine of the SPEC benchmarks it manages to attain an accuracy of 97\% with a 12 bit 512-entry 4-way associative history register table. Tuning the amount of bits in the history registers and entries in the history register table allows a designer to sacrifice performance for size.
%Discuss in a critical way the analysed solutions. For each of them highlight pros and cons. Provide a qualitative (and in case possible also a quantitative) comparison of the discussed approaches. If applicable describe the way you would try to address the problem and explain why do you think your solution could give better results.

\section{Conclusion}
\label{sec:conclusion}
As shown in subsection \ref{ssec:results}, there is a pretty big spread in the benchmark results for the smaller predictors, due to differing methodologies used in the different papers.
For a big predictor the Combination shows the best results by far, with a \SI{98.1}{\percent} accuracy for a predictor with the size of \SI{64}{\kilo\byte}.
For the smaller predictors the results are a little muddier, but Combination still comes out ahead, with \SI{95.2}{\percent} with a size of \SI{0.5}{\kilo\byte}.
The combination predictor used, consists of one local and one gshare predictor.
The gshare predictor gets outperformed by the YAGS predictor, so if we replace the gshare with the YAGS, the Combination predictor might even perform better.
%Summarise the report content and draw the conclusions of your investigation.

% An example of a floating figure using the graphicx package.
% Note that \label must occur AFTER (or within) \caption.
% For figures, \caption should occur after the \includegraphics.
% Note that IEEEtran v1.7 and later has special internal code that
% is designed to preserve the operation of \label within \caption
% even when the captionsoff option is in effect. However, because
% of issues like this, it may be the safest practice to put all your
% \label just after \caption rather than within \caption{}.
%
% Reminder: the "draftcls" or "draftclsnofoot", not "draft", class
% option should be used if it is desired that the figures are to be
% displayed while in draft mode.
%
%\begin{figure}[!t]
%\centering
%\includegraphics[width=2.5in]{myfigure}
% where an .eps filename suffix will be assumed under latex, 
% and a .pdf suffix will be assumed for pdflatex; or what has been declared
% via \DeclareGraphicsExtensions.
%\caption{Simulation Results.}
%\label{fig_sim}
%\end{figure}

% Note that IEEE typically puts floats only at the top, even when this
% results in a large percentage of a column being occupied by floats.


% An example of a double column floating figure using two subfigures.
% (The subfig.sty package must be loaded for this to work.)
% The subfigure \label commands are set within each subfloat command,
% and the \label for the overall figure must come after \caption.
% \hfil is used as a separator to get equal spacing.
% Watch out that the combined width of all the subfigures on a 
% line do not exceed the text width or a line break will occur.
%
%\begin{figure*}[!t]
%\centering
%\subfloat[Case I]{\includegraphics[width=2.5in]{box}%
%\label{fig_first_case}}
%\hfil
%\subfloat[Case II]{\includegraphics[width=2.5in]{box}%
%\label{fig_second_case}}
%\caption{Simulation results.}
%\label{fig_sim}
%\end{figure*}
%
% Note that often IEEE papers with subfigures do not employ subfigure
% captions (using the optional argument to \subfloat[]), but instead will
% reference/describe all of them (a), (b), etc., within the main caption.


% An example of a floating table. Note that, for IEEE style tables, the 
% \caption command should come BEFORE the table. Table text will default to
% \footnotesize as IEEE normally uses this smaller font for tables.
% The \label must come after \caption as always.
%
%\begin{table}[!t]
%% increase table row spacing, adjust to taste
%\renewcommand{\arraystretch}{1.3}
% if using array.sty, it might be a good idea to tweak the value of
% \extrarowheight as needed to properly center the text within the cells
%\caption{An Example of a Table}
%\label{table_example}
%\centering
%% Some packages, such as MDW tools, offer better commands for making tables
%% than the plain LaTeX2e tabular which is used here.
%\begin{tabular}{|c||c|}
%\hline
%One & Two\\
%\hline
%Three & Four\\
%\hline
%\end{tabular}
%\end{table}


% Note that IEEE does not put floats in the very first column - or typically
% anywhere on the first page for that matter. Also, in-text middle ("here")
% positioning is not used. Most IEEE journals/conferences use top floats
% exclusively. Note that, LaTeX2e, unlike IEEE journals/conferences, places
% footnotes above bottom floats. This can be corrected via the \fnbelowfloat
% command of the stfloats package.






% conference papers do not normally have an appendix


% use section* for acknowledgement
%\section*{Acknowledgment}
%
%
%The authors would like to thank...
%




% trigger a \newpage just before the given reference
% number - used to balance the columns on the last page
% adjust value as needed - may need to be readjusted if
% the document is modified later
%\IEEEtriggeratref{8}
% The "triggered" command can be changed if desired:
%\IEEEtriggercmd{\enlargethispage{-5in}}

% references section

% can use a bibliography generated by BibTeX as a .bbl file
% BibTeX documentation can be easily obtained at:
% http://www.ctan.org/tex-archive/biblio/bibtex/contrib/doc/
% The IEEEtran BibTeX style support page is at:
% http://www.michaelshell.org/tex/ieeetran/bibtex/
%\bibliographystyle{IEEEtran}
% argument is your BibTeX string definitions and bibliography database(s)
%\bibliography{IEEEabrv,../bib/paper}
%
% <OR> manually copy in the resultant .bbl file
% set second argument of \begin to the number of references
% (used to reserve space for the reference number labels box)
\begin{thebibliography}{5}

\bibitem{twolevel}
T. Y. Yeh and Y. N. Patt, \emph{Two-Level Adaptive Training Branch Prediction}. Dept. of Elect. Eng. and Comput. Sci., The Univ. of Michigan, Ann Arbor, Michigan, Proc. of the 24th annual international symp. on Microarchitecture, pp. 51–61, Jan. 1991
\bibitem{hybrid}
S. McFarling, \emph{Combining Branch Predictors}. Digital Western Research Laboratory, Palo Alto, CA, Vol. 49. Technical Report TN-36, June 1993 
\bibitem{yags}
A. N. Eden and T. Mudge, \emph{The YAGS Branch Prediction Scheme}, Proceedings of the 31st Annual ACM/IEEE International Symposium on Microarchitecture, MICRO 31, pages 69--77, 1998

% @inproceedings{Eden:1998:YBP:290940.290962,
%  author = {Eden, A. N. and Mudge, T.},
%  title = {The YAGS Branch Prediction Scheme},
%  booktitle = {Proceedings of the 31st Annual ACM/IEEE International Symposium on Microarchitecture},
%  series = {MICRO 31},
%  year = {1998},
%  isbn = {1-58113-016-3},
%  location = {Dallas, Texas, USA},
%  pages = {69--77},
%  numpages = {9},
%  url = {http://dl.acm.org/citation.cfm?id=290940.290962},
%  acmid = {290962},
%  publisher = {IEEE Computer Society Press},
%  address = {Los Alamitos, CA, USA},
% } 

\bibitem{paper3}
Author1 and Author2, \emph{Paper 3 title}, Full reference paper 3. 
\bibitem{paper4}
Author1 and Author2, \emph{Paper 4 title}, Full reference paper 4.
\end{thebibliography}




% that's all folks
\end{document}


