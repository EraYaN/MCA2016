\documentclass[conference]{IEEEtran}

\usepackage{todonotes}
\usepackage[binary-units=true,parse-numbers=true,table-auto-round=true]{siunitx}
\DeclareSIUnit\flops{FLOPS}
\DeclareSIUnit\iops{IOPS}
\sisetup{detect-weight=true}
\usepackage{float}
\usepackage{booktabs}
\usepackage{csquotes}

% correct bad hyphenation here
\hyphenation{op-tical net-works semi-conduc-tor}


\begin{document}
%
% paper title
% can use linebreaks \\ within to get better formatting as desired
% Do not put math or special symbols in the title.
\title{A Comparison of Branch Prediction Strategies}


% author names and affiliations
% use a multiple column layout for up to three different
% affiliations
\author{\IEEEauthorblockN{E.R. de Haan, A.S.J. Misdorp, L.T.J. van Leeuwen}
\IEEEauthorblockA{Computer Engineering Laboratory\\
Delft University of Technology\\
Delft, The Netherlands\\
Email: e.r.dehaan@student.tudelft.nl, a.s.j.misdorp@student.tudelft.nl, l.t.j.vanleeuwen@student.tudelft.nl}
}


% make the title area
\maketitle

% As a general rule, do not put math, special symbols or citations
% in the abstract
\textbf{\textit{Abstract -} Programs can contain a lot of branches. These branches can disrupt the flow of the program and therefore cause a slowdown. In this paper we describe the mechanisms behind four different branch prediction techniques namely: two-level adaptive training, hybrid prediction, YAGS, and neural branch prediction. A final trade-off is made between their accuracies and their relative size-performance ratio based on the data provided by the authors of the branch predictors. This comparison is a little complicated due to the fact that different authors use different benchmarks. However, based on a comparison between relatively similar benchmarks it's shown that hybrid prediction performs the best for large predictor sizes with an accuracy of 98.1\%.}

\IEEEpeerreviewmaketitle

\section{Introduction}

Any arbitrary computer program utilizes several conditional statements like the \textit{if-else} construction and \textit{for} or \textit{while} loops. These conditional statements can prove to be disruptive to the program flow as an instruction is usually fetched from the program counter, where after the next instruction is usually assumed to be at the next address. As current computer architectures employ pipelines, it is usually preferably to fetch the next instruction while the previous one is being executed. However with conditional statements it's uncertain what the next instruction will be. In the case of an \textit{if-else} statement there are two possibilities, both which can only be evaluated once this statement is executed. Normally this would mean that the pipeline would have to stall and wait for the statement to be evaluated. Branch predictors try to minimize the delay caused by potential stalling due to conditional statements. There are various ways to accomplish this, four known methods are: the hybrid predictor, neural branch prediction, two-level adaptive training, and the YAGS branch prediction. These four methods will be described and analyzed after which a comparison will be drawn. With this comparison we would like to answer the following research question:\\\\ \say{Which of the four proposed branch predictors provide the highest accuracy, while requiring the least amount of resources?}\\\\
\todo[inline]{Main conclusion of investigation, perhaps separate paragraph for organisation}
%\begin{itemize}
%\item Introduce the topic and its relevance to computer architecture.
%\item State the Research Question (RQ).
%\item Briefly describe the RQ solutions to be detailed in the report.
%\item State the main conclusions of your investigation.
%\item Report organisation (one paragraph).
%\end{itemize} 


\section{Description of the Evaluated Solutions}
Based on the year of publishing the respective papers are discussed in the following sections.
Therefore section \ref{two-level} starts off with \textit{two-level adaptive training}, published in 1991.
Secondly, in Section \ref{hybrid} the hybrid predictor is explained. In section \ref{ssec:yags} the YAGS predictor is examined while lastly the neural branch predictor is explained in section \ref{sec:neural}.

%The proposed branch prediction schemes of each paper will be explained to serve as context for the quantitative and qualitative comparison that follows this section. Attention will also be paid to general branch prediction concepts, especially in the older papers.

\subsection{Two-level adaptive training}
The first paper that will be discussed is Two-Level Adaptive Training Branch Prediction by Tse-Yu Yeh and Yale N.
Patt \cite{twolevel}.
This paper is the oldest of the four featured in this comparison paper, dating from 1991.
The ideas put forth in this paper are however influential enough to be referred to by two of the other four papers and it is therefore definitely worth discussing.
Two-level adaptive training is a local branch prediction scheme.
The great benefit of local branch prediction is that it is able to remember the histories of multiple conditional branches separately.
This means that the taken or not taken choices of one branch do not necessarily influence the predictions made for other branches in the code.
The two-level adaptive training scheme accomplishes this by using two separate registers: the branch history register table (HRT) and the branch history pattern table (PT).
\subsubsection{Mechanism}
Two-level adaptive training uses pattern repetition to predict the outcome of branches.
The job of the HRT is to save the recent history of choices made in branches.
Ideally every branch has its own register in the table but considering the amount of conditional branches that can be present in a program this is obviously not the case (which will be discussed later).
When a branch is taken a '1' is shifted into the register and if a branch is not taken a '0' will be shifted in.
The contents of the registers in the HRT are used to address the PT.
In that table the actual prediction bits reside.
In order to have separate prediction bits for every possible pattern, the PT will have to have $2^k$ entries where k is the amount of bits in the history registers in the HRT.
The PT has multiple possible schemes it can use to produce predictions for the occurring patterns.
A simple example would be a "last time" scheme where the pattern table simply saves the outcome of the most recent time this pattern occurred and predicts that the outcome will probably be the same this time.
Yeh and Patt favor a saturating counter in their paper as it seemed to perform the best.
This saturating counter is a simple two bit counter that increments whenever a branch is taken and decrements when a branch is not taken.
Saturating means that it won't overflow or underflow past 3 or 0.
The prediction of this saturating counter is then based on it's most significant bit, predicting taken if it is '1' and not taken if it is '0'.
Using such a counter introduces a form inertia in the predictions, meaning that one wrong prediction won't immediately change the next prediction.
Tables of these saturating counters are often called bimodal branch predictors.
\subsubsection{Possible problems}
With a finite amount of resources two-level adaptive training inevitably runs into some limits.
As mentioned before, having an HRT large enough to store a separate history for every single conditional branch in the program is infeasible.
Yeh and Patt suggest two ways of making the two-level adaptive training work with smaller HRT's.
The Associative History Register Table (AHRT) and the Hash History Register Table (HHRT).
The AHRT is a simple set-associative cache where the lower part of a branch address is used to index the table and the higher part is used as a tag.
A least recently used algorithm is used to determine what branch histories are saved.
The AHRT comes with the extra cost of having to store the tag.
The HHRT is a simple hash table so it has the benefit of not having to store the tag, but it has a slightly worse performance because collisions can occur in the table.
This means that the histories of two different branches can interfere with each other.

Another problem is that the two-level adaptive training scheme utilizes only one PT.
This means that if a certain pattern occurs in two different branches with a different outcome predictions will be flawed because the two patterns will address the same prediction bits in the PT.
This interference effect can be reduced by saving a longer history of every branch.

Lastly, there are some possible problems with latency.
Because making a prediction requires two sequential lookups in two different tables, the scheme is somewhat slow.
In order to not have to wait for a prediction Yeh and Patt suggest already determining the next prediction whenever the history register is updated and storing it in the history register for that branch.
This makes the prediction immediately available whenever the HRT is accessed.
Issues can also occur when a new prediction in a branch is required before the last one has been confirmed to be correct or incorrect.
Yeh and Patt suggest to just predict that the branch is taken in such cases as this mostly occurs in very tight loops with a high tendency to be taken.

\subsection{Hybrid predictor}
The first paper that will be discussed is "Combining Branch Predictors" by Scott McFarling \cite{hybrid}. Although the paper dates from 1993 the concept of a hybrid predictor put forth by this paper is still relevant today and it's inclusion in this comparison paper also allows for discussion on some basic categories of branch predictors. Therefore, as is the case in McFarling's paper, before the hybrid predictor is discussed the concept of bimodal, local and global branch prediction will be explained.
\subsubsection{Bimodal branch prediction}
This form of branch prediction is one of the simplest concepts that can be used in order to make somewhat accurate predictions of whether or not a path will be taken in a branch. It uses the recent history of choices made in branches and assumes that the path the branch will take is probably unchanged. Only when it sees it's prediction fail multiple times will it change it's prediction. One could say there is some inertia in it's predictions. To implement bimodal branch prediction one only needs a register of multiple 2 bit counters which we will call the branch history table (BHT). This register can be addressed using the program counter. Each counter in the register corresponds to a different branch in the code. If the path in the prediction is taken the corresponding counter in the BHT is incremented, if the path is not taken the counter is decremented. It is a saturating counter and therefore it will not overflow if an increment or decrement past it's limits is attempted. The prediction made for that particular branch is then based on the most significant bit in the counter. This is what introduces the inertia in the predictions as after a large amount of correct predictions the prediction will only change if there were two bad predictions in a row. The size of the BHT is very important because for small tables multiple program counters of different branches may map to the same address in the BHT causing interference between the histories of those branches and decreasing the accuracy of the branch predictor.
\subsubsection{Local branch prediction}
With this form of branch prediction the concept of pattern repetition is introduced. One can definitely imagine scenarios (for example small program loops) where instead of long repetitions of the same taken or not taken choice, small patterns appear where for example patterns like 1110 with 1 representing taken and 0 representing not taken will repeat a large number of times. With that example pattern, knowing the history of the last three choices will allow for perfect prediction of the next choice. Local branch prediction makes use of this as a second table is introduced which we will call the pattern table (PT). This table is addressed with the program counter and stores the last few choices of every branch. That pattern (for example 1110) is then used to address the BHT that has the counters. This way, every pattern has it's own counter and therefore it's own prediction. Of course, just like with bimodal prediction, the size of the tables is very important. Not only to avoid multiple branches mapping to the same address, but also because the same small pattern in different branches may require a different prediction. A longer saved history for every branch can avoid that hazard. Local prediction is more accurate than bimodal when larger registers in a design are not an issue.
\subsubsection{Global branch prediction}
This concept does away with the history for different branches and instead looks at the recent history of the program itself. It saves the choice made whenever a branch is encountered without care for what branch it is and then uses the resulting pattern to address a register of counters. This of course becomes increasingly accurate with a longer saved history for choices. The paper shows that global branch prediction is worse than local for all register sizes, but better than bimodal for larger register sizes. McFarling also discusses in his paper an expansion on this concept that introduces some locality in the prediction to increase the performance. By concatenating (called gselect in the paper) or XORing (called gshare) the branch address and global history and using that to address a register of counters significantly improved performance is achieved for all register sizes.
\subsubsection{Hybrid branch prediction}
The most important conclusion of McFarling's paper and the reason it's included in this comparison paper is his suggestion for a hybrid predictor. The hybrid predictor functions as a sort of meta-predictor. It combines the concepts mentioned above as two predictors in one and then predicts which of the two is most likely to give a correct prediction. In order to accomplish this it needs an extra register of counters addressed by the program counter. Much in the same way the bimodal branch predictor uses it's counter register it increments or decrements the 2 bit counters in the table depending on which of the two predictors was right and uses the most significant bit in the counter to determine which predictor to trust. Of course, the hybrid predictor allows for multiple combinations of different predictors. In the paper however a focus is put on combining a bimodal or local predictor with a gshare predictor. Using either of these combinations results in higher prediction accuracies than achieved by any of the earlier discussed methods. Using the local/gshare combination seems to obtain the highest achievable accuracy with arrays larger than 2KB approaching 98.1\% accuracy.
%Describe the main idea of the solution in [1] and the way it works. Use your own words and in case really needed you can include figures from the original paper while giving proper reference. The text has to be self contained as the reader should be able to understand the essence of the proposed approach without reading the original paper.

\subsection{YAGS predictor}
\label{ssec:yags}
The third paper that will be discussed is "The YAGS Branch Prediction Scheme" by A. N. Eden and T. Mudge \cite{yags}.
The YAGS predictor is a bimodal predictor at it's core.
It should combat aliasing between biased branches and their instances where they do not agree with their bias.
And lastly it would be beneficial to reduce the amount of unnecessary information in the PHT.

So if the bimodal predictor is used to store the branch bias, one only needs to store the instances of the branches when they do not agree with their bias in the direction PHTs.
This reduces the size of the direction PHTs greatly.
To keep track of those instances a 6 to 8 bit tag is added to the direction PHTs, these tags are based on the least significant bits of the branch address, and so almost all aliasing between two consecutive branches is eliminated.

So when a branch occurs, and the choice PHT indicates a taken bias, the not taken cache is checked to see if there is a cache hit.
If this is the case the cache provides the prediction, otherwise the choice PHT.
The direction caches are updated if one of their predictions was used, and the choice PHT is updated like in a bimodal predictor.
The last way the one of the direction caches is updated is if the choice PHT's supplied prediction turned out to be wrong, the cache of the correct result is updated, because the branch did not agree with it's bias there.

A last improvement is to make the direction caches set-associative, to reduce the aliasing even further.
A LRU (least recently used) replacement policy is used, with the exception that an entry in the taken cache indicating not taken will be replaced first to avoid unnecessary information, this information is already in the choice PHT and therefore redundant.

\subsection{Neural Branch prediction}

In 2003 Jiminez \cite{neural} proposed a modification of the standard neural branch predictors called \enqoute{Fast Path-Based Neural Branch Prediction}. The main reason for this modification is that a regular neural branch predictor has high latency. Jiminez managed to decrease this latency by spreading the work required for neural branch prediction over time. First a brief introduction to neural branch prediction is given after which the Fast Path-based version is described.\\

Neural branch prediction makes use of perceptrons. A perceptron is based on a neuron which has dendrites to receive information, a cell body to process this information, and axon terminals to relay the output. A perceptron is highly similar in the sense that it has inputs, a processing stage, and an output. The inputs of a perceptron take in previous branch result that represent either taken or not taken. These branch results are stored in a history register. Every input of a perceptron is weighted, this way a potential correlation between inputs can be indicated.  When a branch prediction has to be made, the branch address is hashed to select a perceptron from a table. Subsequently the dot product of the input, the history register, and the corresponding integer weights of the perceptron is taken. This outputs a single integer that represents the output of the perceptron. A positive output indicates that the branch should be taken, while a negative output indicates the branch shouldn't be taken. The key to successful prediction is that the perceptrons are trained. Once the branch result is know the perceptron is confronted with its prediction from which it can learn. Subsequently the perceptron's weights are updated to improve the prediction.

\subsubsection{Fast Path-Based Neural Branch Prediction}
The main problem with regular neural branch prediction is latency. Normally a perceptron is selected based on the hash of the branch address after which the output is calculated as the dot product of the weights and the history vector. A path-based algorithm works differently in that the weights are chosen along the path to the branch. Take for example a regular perceptron predictor with six weights that are used to predict branch $B_t$. Instead of taking the dot product of all the weights with the history register at once, five of the six weights are added to its corresponding value in the history register during the prediction of the previous branches $B_6$ until $B_1$. These additions are summed and during the prediction of branch $B_t$, the last weight and history register addition is performed and added to the sum. This method where the computation is spread out over time has lead to a 16\% higher IPC on a predictor with 64 KB hardware. Another positive effect is that this method increases accuracy as the path history is taken into account. Lastly the weights can be updated again according to the actual branch result.

% metric is SPEC95 branch prediction correctness (between 0-1) Quantitative
% metric is size on-chip Qualitative

\section{Comparison}
\subsection{Metrics}
As a comparison metric the branch prediction accuracy is the most important specification. This specification can be derived via the SPEC benchmark, which is displayed for every branch predictor in tables \ref{tab:spec-accuracy-big} and \ref{tab:spec-accuracy-small}. The accuracy of the difference predictors is shown, note that some of these values were retrieved from a figure so there is a margin of error. It should also be noted that SPEC benchmarks are updated over time and so some of the branch predictors are compared using different SPEC benchmarks.\\A second important metric we will discuss is the size of each of the solutions and their relative performance.
\subsection{Results}
\label{ssec:results}
\begin{table}[H]
    \centering
    \caption{SPEC benchmarks accuracy result for big predictor sizes.}
    \label{tab:spec-accuracy-big}
    \begin{tabular}{llS[table-format=3.1,table-space-text-post=\si{\kilo\byte}]S[table-format=1.3]}
    \toprule
            {\textbf{Predictor}} & {\textbf{Benchmarks}} & {\textbf{Size ($\approx$)}} & {\textbf{Accuracy}} \\
        \midrule
            {Combined (local/gshare)} & SPEC 89 & 64\si{\kilo\byte} & 0.981 \\
            {Two-level} & SPEC 89 & 14\si{\kilo\byte} & 0.97 \\
            {Standard Neural} & SPEC 95 + 00 (int) & 64\si{\kilo\byte} & 0.939 \\
            {Fast-path Neural} & SPEC 95 + 00 (int) & 64\si{\kilo\byte} & 0.942 \\
            {gshare} & SPEC95 & 38\si{\kilo\byte}  & 0.945 \\
            {bi-mode} & SPEC95 & 35\si{\kilo\byte}  & 0.950 \\
            {YAGS} & SPEC95 & 48\si{\kilo\byte}  & 0.955 \\
        \bottomrule
    \end{tabular}
\end{table}

\begin{table}[H]
    \centering
    \caption{SPEC benchmarks accuracy result for small predictor sizes.}
    \label{tab:spec-accuracy-small}
    \begin{tabular}{llS[table-format=3.1,table-space-text-post=\si{\kilo\byte}]S[table-format=1.3]}
        \toprule
            {\textbf{Predictor}} & {\textbf{Benchmarks}} & {\textbf{Size ($\approx$)}} & {\textbf{Accuracy}} \\
        \midrule
            {Combined (local/gshare)} & SPEC 89 & 0.5\si{\kilo\byte} & 0.952 \\
            {Two-level} & N/A & & \\
            {Standard Neural} & SPEC 95 + 00 (int) & 1\si{\kilo\byte} & 0.912 \\
            {Fast-path Neural} & SPEC 95 + 00 (int) & 1\si{\kilo\byte} & 0.925 \\
            {gshare} & SPEC95 & 0.5\si{\kilo\byte} & 0.69 \\
            {bi-mode} & SPEC95 & 0.5\si{\kilo\byte} & 0.73 \\
            {YAGS} & SPEC95 & 0.5\si{\kilo\byte} & 0.77 \\
        \bottomrule
    \end{tabular}
\end{table}

\subsubsection*{Combined}
The combined branch predictor is benchmarked using SPEC 89, running the programs for 10 million instructions.
For large predictor sizes, around 64 KB, table \ref{tab:spec-accuracy-big} shows that the accuracy of the best performing predictor combination appears to be two-level/gshare as it approaches 98.1\%. This is better than either of the two alone. Decreasing the size of the predictor leads to less accuracy as shown in table \ref{tab:spec-accuracy-small}. However the combined predictor still has the best prediction accuracy. A great advantage of the combined predictor is that it's very adaptable. There are a lot of combinations of predictors that could possibly be implemented and if resources are at a premium, the predictor can easily be shrunk at the cost of performance.

\subsubsection*{Two-level}
On nine of the SPEC benchmarks two-level adaptive training manages to attain an average accuracy of 97\% with a 12 bit 512-entry 4-way associative history register table with a pattern table comprised of saturating counters. 
The size of the two-level predictor that is mentioned in the table is calculated as the addition of the size of the HRT (512 entries of 12 bits) and the PT ($2^{12}$ entries of 2 bits). 
The data in Yeh and Patt's paper doesn't lend itself to a good comparison based on predictor size which is why the two-level entry in the small predictor table is left blank \cite{twolevel}. 
Adjusting the amount of bits in the history registers and the amount of history registers in the HRT still do allow for a predictor that is very adjustable in size.


\subsubsection*{Standard Neural, Fast-path Neural}
The neural predictors are mainly simulated with the SPEC CPU 2000 integer benchmark next to some SPEC 95 integer benchmarks that are not duplicated in SPEC 2000. From table \ref{tab:spec-accuracy-big} it can be seen that the standard neural branch prediction and the fast-path neural branch prediction are not among the best. Looking at table \ref{tab:spec-accuracy-small} however, the predictors are now the second best options. In both cases is the fast-path based algorithm faster than the standard implementation. The parallelization of the algorithm proves to be beneficial. Decreasing the size has little influence on the prediction accuracy with about a difference of 2 percent point. Similarly however this means that increasing the size will not lead to much better accuracy. The fast-path based algorithm takes the path history into account but this influence levels off as the size increases.

\subsubsection*{gshare, bi-mode, YAGS}
gshare, bi-mode, and YAGS predictors were simulated using all 8 SPEC 95 benchmarks interleaved each 60 000 instructions to simulate a high context switch environment.
This is probably also why the accuracy of these algorithms is so poor compared to the other paper's figures.

YAGS performs much better when it is scaled down compared to the gshare and bi-mode predictors \cite{yags}.
This is because the YAGS prediction contains less useless information in it's tables, as described in subsection \ref{ssec:yags}.
So you can get the same performance with less resources.

Eden and Mudge believe that they have not shown the full potential of this predictor scheme.
Due to the relatively small size of the direction caches, the history register is also quite small, hence the predictor has reduced correlation information.
A proposed improvement is to add those lost history bits as tags.
This can be done for all the predictors in the paper, but they stated that the overhead would be much smaller for the YAGS prediction scheme \cite{yags}. 

%Discuss in a critical way the analysed solutions. For each of them highlight pros and cons. Provide a qualitative (and in case possible also a quantitative) comparison of the discussed approaches. If applicable describe the way you would try to address the problem and explain why do you think your solution could give better results.

\section{Conclusion}
\label{sec:conclusion}
As shown in subsection \ref{ssec:results}, there is a pretty big spread in the benchmark results for the smaller predictors, due to differing methodologies used in the different papers.
Therefore making quantitative comparisons a bit harder.
For a big predictor the combined variety shows the best results, with a \SI{98.1}{\percent} accuracy for a predictor with the size of \SI{64}{\kilo\byte}.
For the smaller predictors the results are a little muddier, but Combination still comes out ahead, with \SI{95.2}{\percent} with a size of \SI{0.5}{\kilo\byte}.
The combined predictor used, consists of one local and one gshare predictor.
As shown in the figures \ref{fig:performance-norm-big} and \ref{fig:performance-norm-small}, the gshare predictor gets outperformed by the YAGS predictor, so if we replace the gshare with the YAGS, the combined predictor might perform even better.

\begin{thebibliography}{5}

\bibitem{twolevel}
T. Y. Yeh and Y. N. Patt, \emph{Two-Level Adaptive Training Branch Prediction}. Dept. of Elect. Eng. and Comput. Sci., The Univ. of Michigan, Ann Arbor, Michigan, Proc. of the 24th annual international symp. on Microarchitecture, pp. 51–61, Jan. 1991
\bibitem{hybrid}
S. McFarling, \emph{Combining Branch Predictors}. Digital Western Research Laboratory, Palo Alto, CA, Vol. 49. Technical Report TN-36, June 1993 
\bibitem{yags}
A. N. Eden and T. Mudge, \emph{The YAGS Branch Prediction Scheme}, Proceedings of the 31st Annual ACM/IEEE International Symposium on Microarchitecture, MICRO 31, pages 69--77, 1998
\bibitem{neural}
Jim{\'e}nez, Daniel A, \emph{Fast path-based neural branch prediction}, Microarchitecture, 2003. MICRO-36. Proceedings. 36th Annual IEEE/ACM International Symposium on, IEEE, pages 243--252, 2003 


\end{thebibliography}

\end{document}