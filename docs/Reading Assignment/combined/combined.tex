\subsection{Hybrid predictor}
The second paper that will be discussed is "Combining Branch Predictors" by Scott McFarling \cite{hybrid}. With his idea for a combined predictor McFarling combines older branch prediction concepts into a new form with increased accuracy. Before McFarling goes into detail for his hybrid predictor he first discusses some basic branch prediction concepts of which some are already discussed in the last section. Recall, bimodal prediction is the use of saturating counters to make predictions that don't immediately change upon an incorrect choice and local prediction is the idea that one saves separate histories for different conditional branches for more accurate case specific predictions. The local prediction scheme McFarling mentions is in fact the very same two-level adaptive training scheme discussed in the last section. In order to gain a better understanding of McFarling's predictor, global branch prediction will briefly have to be explained.
\subsubsection{Global branch prediction}
This concept does away with the history for different branches that local branch prediction utilises and instead looks at the recent history of the program itself. It saves the choice made whenever a branch is encountered without care for what branch it is and then uses the resulting pattern to address a register of counters. This of course becomes increasingly accurate with a longer saved history for choices. The paper shows that global branch prediction is worse than local for all register sizes, but better than bimodal for larger register sizes. McFarling also discusses in his paper an expansion on this concept that introduces some locality in the prediction to increase the performance. By concatenating parts of (called gselect in the paper) or XORing (called gshare) the branch address and global history and using that to address a register of counters significantly improved performance is achieved for all register sizes. Doing this essentially reintroduces a certain degree of locality to the prediction scheme as the branch address start influencing the prediction again. Global branch prediction can be especially effective when the outcome of a branch depends on the outcome of other recent branches.
\subsubsection{Hybrid branch prediction}
The most important conclusion of McFarling's paper and the reason it's included in this comparison paper is his suggestion for a hybrid predictor. The hybrid predictor functions as a sort of meta-predictor. It combines the concepts mentioned above as two predictors in one and then predicts which of the two is most likely to give a correct prediction. In order to accomplish this it needs an extra register of counters addressed by the program counter. Much in the same way the bimodal branch predictor uses its counter register it increments or decrements the 2 bit counters in the table depending on which of the two predictors was right and uses the most significant bit in the counter to determine which predictor to trust. Of course, the hybrid predictor allows for multiple combinations of different predictors. In the paper however a focus is put on combining a bimodal or local predictor with a gshare predictor. Using either of these combinations results in higher prediction accuracies than achieved by any of the earlier discussed methods. Using the local/gshare combination seems to obtain the highest achievable accuracy with arrays larger than 2KB approaching 98.1\% accuracy.